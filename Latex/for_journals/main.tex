\documentclass[12pt,letterpaper]{IEEEconf}
\usepackage{titling}
\title{\LARGE Techniques Of Electromagnetic Cloaking}
\author{Gutierrez Benito, Roly Sandro}

\begin{document}
\maketitle
\renewcommand{\section}[1]
{\addtocounter{section}{1}\begin{center}\arabic{section}. #1
\end{center}}

\thispagestyle{empty}
\pagestyle{empty}

\begin{abstract}\bf
	How to conceal objects from electromagnetic radiation has 
	been a hot research topic. Radar is an object detection 
	system that uses Radio waves to determine the range, angle,
	or velocity. A radar transmit radio waves or microwaves that 
	reflect from any object in their path. A receive radar is
	typically the same system as transmit radar, receives and
	processes these reflected wave to determine properties of
	object. Different organizations are working onto hide object
	from the radar in outer space. Any confidential object can be
	taken through space without being detected by the enemies.
	This calls for necessity of devising new method to conceal 
	an object electromagnetically.
\end{abstract}

\section{INTRODUCTION}
The first electromagnetic cloaking device was produced in 2006,
using gradient-index meta-materials. This has led to the 
burgeoning field of transformation optics (and now 
transformation acoustics), where the propagation of waves is
precisely manipulated by controlling the behavior of the
material through which the light (sound) is travelling.
One of the ways to cloak an object by manipulating the
electromagnetic waves around is to cover the object with an
anisotropic and inhomogeneous object(called a cloak) that bends
the incident wave which essentially hides the object concerned.
This comes at a great cost though. The electrical and magnetic
characteristics, such as permittivity and permeability, of such
a cloak approach extremes. The preceding method belongs to a 
category known as transformation optics and one of the drawbacks
of transformation optics is its vulnerability to pulses because 
no cloak till date has responded well to a pulse.
This  combined with the ideal electromagnetic parameters required
for a cloak along with the electrical losses in the thick shells
of the cloak makes use of such a material a non-viable option and
thus a need to come up with something else.

\end{document}

