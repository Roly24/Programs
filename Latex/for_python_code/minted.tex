%%%%%%%%%%%%%%%%%%%%%%%% Preambule %%%%%%%%%%%%%%%%%%%%%%%%%%%%%
\documentclass{article}
\usepackage{minted}
\usepackage{xcolor}
\usepackage{amsmath}
\usemintedstyle{monokai}
\parindent = 0mm

\newenvironment{pycode}
	{\VerbatimEnvironment
	\begin{minted}[
	xleftmargin=10pt,
	linenos=true,
	mathescape,
	frame=lines,
	framesep=2mm,
	obeytabs=true,
	tabsize=4,
	bgcolor=black
	]{python}}
	{\end{minted}}

\newcommand{\pyfile}[1]{\inputminted[
	xleftmargin=10pt,
	linenos=true, 
	frame=lines,
	framesep=2mm,
	tabsize=4,
	bgcolor=black
	]{python}{#1}}
%%%%%%%%%%%%%%%%%%%%%%%% Document %%%%%%%%%%%%%%%%%%%%%%%%%%%%%%
\begin{document}
 \begin{pycode}
class Test:
	def __init__(self, *args, **kwargs):
		super().__init__(*args, **kwargs)
		self.geometry("300x200")

	def button():
		for i in range(10):
			print("This is just a test")

object1 = Test()
  \end{pycode}

For a single line of source code, you can alternatively use a 
shorhand notation: 
\mint{python}|import this|
 
For \textbf{inline use} we write \verb|\mintinline| use command:
\mintinline{python}{print(x**2)}. This works!. 
 
For read and format whole files. Its syntax is \verb|\inputminted|:
\pyfile{hello.py}

Here's another example: we want to use the \LaTeX math mode inside 
comments:

\begin{pycode}
# Returns $\displaystyle\sum_{i=1}^{n}i$
def sum_from_one_to(n):
	r = range(1, n + 1)
	return sum(r)
\end{pycode}

\end{document}

