\documentclass{standalone}
\usepackage{siunitx}
\usepackage{circuitikz}
\begin{document}
\begin{circuitikz}
	\ctikzset{logic ports/scale=0.4}
\ctikzset{logic ports=ieee}
	\draw[step=1cm, gray!50, very thin] (-0.9,-0.9) grid(13,13);
	\foreach \x in {0,...,13}
	{
		\draw[black!50] (\x cm, 1pt) -- (\x cm, -1pt) 
		node[anchor=north]{$\x$};
	}
	\foreach \y in {0,...,13}
	{
		\draw[black!50] (1pt,\y cm) -- (-1pt,\y cm) 
		node[left]{$\y$};
	}


	\ctikzset{multipoles/thickness=2}
	\ctikzset{multipoles/external pins thickness=1}
	\tikzset{mux 2by1/.style={muxdemux,
			muxdemux def={NL=2, NB=0, 
			Lh=1, Rh=0.5, w=1, square pins=1}}}

	\draw (5,5) node[dipchip,
					num pins=6,
					hide numbers,
					external pins width=0.3,
					external pad fraction=4 ](C){};
	\draw[line width=0.4]
		(5,5.15) node[mux 2by1](B){\tiny MUX};

	\draw
		(C.bpin 1) -| (B.lpin 1) 
		(C.bpin 2) --(B.lpin 2)
		(B.rpin 1) |- (C.bpin 5);

		\draw[scale=0.5] (C.pin 3) -|++(-1,-0.3) 
		to[R] ++ (0,-3) node[ground]{};
	\draw (C.pin 6) -|++(0.5,0.5) node[vcc]{+5V};
	% --------------------------------------------------------
	\foreach \pinright in {1,2,3}
	{
		\draw node[right,font=\tiny] at(C.pin \pinright) 
		{$\pinright$};
	}
	\foreach \pinleft in {4,5,6}
	{
		\draw node[left,font=\tiny] at (C.pin \pinleft) 
		{$\pinleft$};
	}
	\node[font=\tiny] at (5,4.3){\text{\si{\raiseto{2}{R}}}};
\end{circuitikz}
\end{document}
